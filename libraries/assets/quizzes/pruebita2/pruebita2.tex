\documentclass[11pt]{article}
\usepackage{tipa}
\usepackage[utf8]{inputenc}
\usepackage[T1]{fontenc}

\usepackage[hmargin=1cm,vmargin=1cm]{geometry}
\geometry{a4paper}

\usepackage{fancyhdr} % This should be set AFTER setting up the page geometry
\pagestyle{fancy} % options: empty , plain , fancy
\renewcommand{\headrulewidth}{0pt} % customise the layout...
\lhead{}\chead{}\rhead{}
\lfoot{}\cfoot{}\rfoot{}


\begin{document}

\noindent \textbf{D\'e la descripci\'on articulatoria completa o nombre el fonema según haga falta:} 

\begin{enumerate}
	\item \textipa{/w/}
	\item \textipa{/\textteshlig/}
	\item \textipa{oclusivo, bilabial, sordo}
	\item \textipa{alta, anterior, estirada}
	\item \textipa{nasal, palatal, sonoro}
\end{enumerate}

\noindent \textbf{Usted es un lingüista atrevido. Acaba de viajar a una aldea perdida del 
sur de Francia y ha descubierto una lengua nueva. Conteste las siguientes preguntas en base a los 
datos que acaba de transcribir (Pista: la ortografía no le va a ayudar).}

\vspace{.2in}

\begin{center}
\begin{tabular}{@{}llll@{}}
\hline \\ [-.5em]
Ortograf\'ia & Transcripci\'on fon\'emica & Transcripci\'on fon\'etica                    & Significado \\ [.5em]
\hline \\ [-.5em]
`balupa'     & \textipa{/ba.lu.\textprimstress pa/} & \textipa{[ba.lu.\textprimstress pa]}          & Persona fea de avanzada edad. \\ [.5em]
`palupa'     & \textipa{/\textprimstress ba.lu.pa/} & \textipa{[\textprimstress p\super{h}a.lu.pa]} & Persona fea de avanzada edad. \\ [.5em]
`balupe'     & \textipa{/ba.lu.\textprimstress pe/} & \textipa{[ba.lu.\textprimstress pe]}          & Persona joven e inteligente. \\ [.5em]
\hline
\end{tabular}
\end{center}

\begin{enumerate}
	\item ¿Cu\'antos al\'ofonos tiene el fonema /b/?
	\item \textipa{/e/} es un fonema (i.e. es contrastivo) en esta lengua, ¿Sí o no?
\end{enumerate}

\noindent \textbf{Bonus} Confirme la validez de la siguiente afirmación: El profesor Joseph es balupe. 




\vspace{.5in}




\noindent \textbf{D\'e la descripci\'on articulatoria completa o nombre el fonema según haga falta:} 

\begin{enumerate}
	\item \textipa{/w/}
	\item \textipa{/\textteshlig/}
	\item \textipa{oclusivo, bilabial, sordo}
	\item \textipa{alta, anterior, estirada}
	\item \textipa{nasal, palatal, sonoro}
\end{enumerate}

\noindent \textbf{Usted es un lingüista atrevido. Acaba de viajar a una aldea perdida del 
sur de Francia y ha descubierto una lengua nueva. Conteste las siguientes preguntas en base a los 
datos que acaba de transcribir (Pista: la ortografía no le va a ayudar).}

\vspace{.2in}

\begin{center}
\begin{tabular}{@{}llll@{}}
\hline \\ [-.5em]
Ortograf\'ia & Transcripci\'on fon\'emica & Transcripci\'on fon\'etica                    & Significado \\ [.5em]
\hline \\ [-.5em]
`balupa'     & \textipa{/ba.lu.\textprimstress pa/} & \textipa{[ba.lu.\textprimstress pa]}          & Persona fea de avanzada edad. \\ [.5em]
`palupa'     & \textipa{/\textprimstress ba.lu.pa/} & \textipa{[\textprimstress p\super{h}a.lu.pa]} & Persona fea de avanzada edad. \\ [.5em]
`balupe'     & \textipa{/ba.lu.\textprimstress pe/} & \textipa{[ba.lu.\textprimstress pe]}          & Persona joven e inteligente. \\ [.5em]
\hline
\end{tabular}
\end{center}

\begin{enumerate}
	\item ¿Cu\'antos al\'ofonos tiene el fonema /b/?
	\item \textipa{/e/} es un fonema (i.e. es contrastivo) en esta lengua, ¿Sí o no?
\end{enumerate}

\noindent \textbf{Bonus} Confirme la validez de la siguiente afirmación: El profesor Joseph es balupe. 


\end{document}