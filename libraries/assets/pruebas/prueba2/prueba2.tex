%
%  Created by Joseph Casillas on 2011-09-15.
%  Copyright (c) 2011 . All rights reserved.
%
%
\documentclass[12pt]{exam}
\usepackage{listings}
\usepackage{pdfsync}
\usepackage{ling}
\usepackage{multicol}
\usepackage{wasysym}
\usepackage{booktabs}
\usepackage{tipa}

\setlength{\topmargin}{-0.1in}
\usepackage{sectsty}
\sectionfont{\normalsize}


\firstpageheader{\begin{bf} Fonética y español oral -- Casillas\\ Prueba II\end{bf}} {} {\begin{bf}Alumno: \rule{33.65mm}{.3mm} \\ Calificación:\rule{20mm}{.3mm} /100\end{bf}}
\footer{\begin{bf} Otoño -- 2015\end{bf}} {} {\begin{bf}Puntos: \rule{31.65mm}{.3mm}\end{bf}}
\runningheader{\begin{bf}Fonética y español oral\end{bf}}{}{\begin{bf}Prueba II\end{bf}}
\addpoints


\begin{document}

\vspace{0.1in}

\section{Escriba el fonema que corresponde a la descripción. (10 puntos): \_\_\_\_\_\_\_\_/10}
\renewcommand{\arraystretch}{1.75}
	\begin{tabular}{@{}lll@{}}
	1.  & \_\_\_\_\_\_\_\_\_\_\_\_\_\_\_\_\_\_\_\_\_\_\_ & oclusivo bilabial sonoro    \\
	2.  & \_\_\_\_\_\_\_\_\_\_\_\_\_\_\_\_\_\_\_\_\_\_\_ & oclusivo dental sordo       \\
	3.  & \_\_\_\_\_\_\_\_\_\_\_\_\_\_\_\_\_\_\_\_\_\_\_ & fricativo velar sordo       \\
	4.  & \_\_\_\_\_\_\_\_\_\_\_\_\_\_\_\_\_\_\_\_\_\_\_ & nasal palatal sonoro        \\
	5.  & \_\_\_\_\_\_\_\_\_\_\_\_\_\_\_\_\_\_\_\_\_\_\_ & africado alveopalatal sordo \\
	6.  & \_\_\_\_\_\_\_\_\_\_\_\_\_\_\_\_\_\_\_\_\_\_\_ & lateral alveolar sonoro     \\
	7.  & \_\_\_\_\_\_\_\_\_\_\_\_\_\_\_\_\_\_\_\_\_\_\_ & alta anterior sonora        \\
	8.  & \_\_\_\_\_\_\_\_\_\_\_\_\_\_\_\_\_\_\_\_\_\_\_ & media posterior sonora      \\
	9.  & \_\_\_\_\_\_\_\_\_\_\_\_\_\_\_\_\_\_\_\_\_\_\_ & fricativo alveolar sordo    \\
	10.	& \_\_\_\_\_\_\_\_\_\_\_\_\_\_\_\_\_\_\_\_\_\_\_ & vibrante múltiple sonoro    \\
	\end{tabular}

\section{Describa los siguientes fonemas (30 pts): \_\_\_\_\_\_\_\_/30}

	\begin{center}
		\begin{tabular}{|c|c|c|c|}
			\hline
			    & Modo de articulación & Punto de articulación & Sonoridad (sordo/sonoro)\\
			\hline
			\textipa{/g/} & & & \\
			\hline
			\textipa{/x/} & & & \\
			\hline
			\textipa{/r/} & & & \\
			\hline
			\textipa{/n/} & & & \\
			\hline
			\textipa{/u/} & & & \\
			\hline
			\textipa{/d/} & & & \\
			\hline
			\textipa{/m/} & & & \\
			\hline
			\textipa{/a/} & & & \\
			\hline
			\textipa{/f/} & & & \\
			\hline
			\textipa{/w/} & & & \\
			\hline
		\end{tabular}
	\end{center}



\section{Escriba la respuesta correcta en el espacio a la izquierda de la oración. (30 puntos): \_\_\_\_\_\_\_\_/30}

\renewcommand{\arraystretch}{1.75}
	\begin{tabular}{@{}llp{2cm}@{}}
	 1.  & \_\_\_\_\_\_\_\_\_\_\_\_\_\_\_\_\_\_ & Un sonido que se produce sin vibración de las cuerdas vocales se \\&& denomina\ldots        \\
	 2.  & \_\_\_\_\_\_\_\_\_\_\_\_\_\_\_\_\_\_ & Un ejemplo de un articulador activo es\ldots                                          \\
	 3.  & \_\_\_\_\_\_\_\_\_\_\_\_\_\_\_\_\_\_ & Un fonema producido con algún tipo de obstrucción del paso del aire \\&& es una\ldots       \\
	 4.  & \_\_\_\_\_\_\_\_\_\_\_\_\_\_\_\_\_\_ & Un ejemplo de articulador pasivo es\ldots \\
	 5.  & \_\_\_\_\_\_\_\_\_\_\_\_\_\_\_\_\_\_ & ¿Qué tienen en común las siguientes consonantes /k x s f/? \\&& Todas son \_\_\_\_\_\_.\\
	 6.  & \_\_\_\_\_\_\_\_\_\_\_\_\_\_\_\_\_\_ & El modo de articulación en el cual se produce un obstáculo total y \\&& luego un obstáculo parcial se llama \_\_\_\_\_\_.\\
	 7. & \_\_\_\_\_\_\_\_\_\_\_\_\_\_\_\_\_\_ & Las vocales altas son \_\_\_\_\_\_ y \_\_\_\_\_\_.\\
	 8. & \_\_\_\_\_\_\_\_\_\_\_\_\_\_\_\_\_\_ & ¿Qué tiene en común los siguientes fonemas: /s, r, l, n/?\\
	 9. & \_\_\_\_\_\_\_\_\_\_\_\_\_\_\_\_\_\_ & Cuándo dos vocales contiguas están en dos sílabas diferentes se dice \\&& que hay un \_\_\_\_\_\_.\\
	 10. & \_\_\_\_\_\_\_\_\_\_\_\_\_\_\_\_\_\_ & ¿Cuál es el elemento obligatorio de una sílaba?\\
	 11. & \_\_\_\_\_\_\_\_\_\_\_\_\_\_\_\_\_\_ & ¿Cuál es la estructura silábica preferida en español?\\
	 12. & \_\_\_\_\_\_\_\_\_\_\_\_\_\_\_\_\_\_ & Provee un ejemplo de una sílaba que contenga coda y subráyala \\
	 13. & \_\_\_\_\_\_\_\_\_\_\_\_\_\_\_\_\_\_ & Provee un ejemplo de un grupo consonántico prohibido en español. \\
	 14. & \_\_\_\_\_\_\_\_\_\_\_\_\_\_\_\_\_\_ & ¿Qué tienen en común los siguientes fonemas consonánticos: /j, ñ/? \\
	 15. & \_\_\_\_\_\_\_\_\_\_\_\_\_\_\_\_\_\_ & Provee un ejemplo de una palabra que contenga una sílaba abierta y \\&& subraya la sílaba\\
	\end{tabular}

\section{Conteste las siguientes preguntas (10 ptos): \_\_\_\_\_\_\_\_/10}
	\begin{questions}
		\question ¿Cómo se sabe si un sonido es sordo o sonoro? Explique las diferencias fundamentales entres estos dos clases de sonidos. 
		\fillwithlines{2in}
		\question Explique las diferencias entre la descripción articulatoria de las vocales frente a la de las consonantes. ¿Qué criteria se usa en cada caso? ¿Qué tienen en común las consonantes y las vocales y cómo son diferentes?
		\fillwithlines{2in}
		\question Explique la diferencia entre \textbf{fonema} y \textbf{alófono}. ¿Cómo se sabe de cuál se trata? Dé ejemplos.
		\fillwithlines{2in}
	\end{questions}


\section{Escriba una transcripción fonológica (\textipa{/fo.no.\textprimstress lo.xi.ka/}) de las siguientes oraciones. Incluye los fonemas, encadenamientos, diptongos/hiatos y separe en sílabas (20ptos): \_\_\_\_\_\_\_\_/20 }

	\begin{questions}
		\question Voy a ir a la tienda hindú del barrio italiano a comprar un vestido.
		\fillwithlines{.5in}
		\question A Juana la eligieron reina del concurso de belleza internacional.
		\fillwithlines{.5in}
		\question Juan Andrés es dueño de una huerta grande en la ciudad de Medellín.
		\fillwithlines{.5in}
		\question Raquel estuvo esperando todo el día en frente del mismo banco. 
		\fillwithlines{.5in}
		\question Al oír esos gritos uno se da cuenta de que el corazón de toda la hinchada ha entrado en el juego. 
		\fillwithlines{.5in}
	\end{questions}



\section{Puntos extras (3 como máximo): \_\_\_\_\_\_\_\_/3 }
\begin{questions}
	\question Describa detalladamente la composición de la africada \textipa{/\textteshlig/}. ¿Qué tiene de peculiar con respecto a los demás modos de articulación? ¿Existen otras africadas en español? ¿Cuáles? ¿Cómo se diferencian con respecto a \textipa{/\textteshlig/}?
	\fillwithlines{2in}
\end{questions}






	
\end{document}