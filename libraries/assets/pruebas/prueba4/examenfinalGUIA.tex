%
%  Created by Joseph Casillas on 2011-09-15.
%  Copyright (c) 2011 . All rights reserved.
%
%
\documentclass[12pt]{exam}
\usepackage{listings}
\usepackage{pdfsync}
\usepackage{ling}
\usepackage{multicol}
\usepackage{wasysym}
\usepackage{booktabs}

\setlength{\topmargin}{-0.1in}
\usepackage{sectsty}
\sectionfont{\normalsize}


\firstpageheader{\begin{bf} Fonética y español oral -- Casillas\\ Prueba 4\end{bf}} {} {\begin{bf}Alumno: \rule{33.65mm}{.3mm} \\ Calificación:\rule{20mm}{.3mm} /100\end{bf}}
\footer{\begin{bf} Otoño -- 2015\end{bf}} {} {\begin{bf}Puntos: \rule{31.65mm}{.3mm}\end{bf}}
\runningheader{\begin{bf}Fonética y español oral\end{bf}}{}{\begin{bf}Prueba 4\end{bf}}
\addpoints

\begin{document}

\vspace{0.1in} 


\section{Escribe la respuesta correcta en el espacio a la izquierda de la oración. (30 puntos): \_\_\_\_\_\_\_\_/30}

\renewcommand{\arraystretch}{1.95}
	\begin{tabular}{@{}llp{2cm}@{}}
	 1.  & \_\_\_\_\_\_\_\_\_\_\_\_\_\_\_\_\_\_ & Un sonido que se produce sin vibración de las cuerdas vocales se \\&& denomina        \\
	\end{tabular}

\vspace{.4in}

\section{Contesta las siguientes preguntas. Da todo el detalle posible (30 puntos): \_\_\_\_\_\_\_\_/30}

\begin{itemize}
	\item 6 preguntas, 5 puntos cada una
	\item Posibles temas:
	\begin{itemize}
		\item Zonas dialectales (y sus explicaciones históricas)
		\item Variación dialectal (fenómenos específicos, contextos fónicos, etc.)
		\item Procesos fonológicos del español estándar
		\item Fonología teórica
	\end{itemize}
\end{itemize}

\vspace{.4in}

\section{Escribe una transcripción /fonológica/ y [fonética] de las siguientes oraciones. Incluye los fonemas o alófonos (si es necesario), los encadenamientos y separa en sílabas. Escribe en el margen de qué variedad del español se trata. (30 puntos): \_\_\_\_\_\_\_\_/30}

\begin{itemize}
	\item 3 oraciones, hay que especificar una variedad (la que quieras)
\end{itemize}

\vspace{.4in}

\section{Decide si las siguientes transcripciones aportan ejemplos de seseo, ceceo o distinción. (10 puntos): \_\_\_\_\_\_\_\_/10}

\begin{itemize}
	\item Os pongo una lista de palabras junto con sus respectivas transcripciones.
	\item Tenéis que decidir si se trata de un ejemplo de seseo, ceceo o distinción.
\end{itemize}





\end{document}