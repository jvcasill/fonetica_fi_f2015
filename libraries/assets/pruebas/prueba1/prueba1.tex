%
%  Created by Joseph Casillas on 2015-09-15.
%  Copyright (c) 2015 . All rights reserved.
%
%
\documentclass[12pt]{exam}
\usepackage{listings}
\usepackage{pdfsync}
\usepackage{ling}
\usepackage{multicol}
\usepackage{wasysym}
\usepackage{booktabs}

\setlength{\topmargin}{-0.1in}
\usepackage{sectsty}
\sectionfont{\normalsize}


\firstpageheader{\begin{bf} Fonética y español oral -- Casillas\\ Prueba I\end{bf}} {} {\begin{bf}Alumno: \rule{33.65mm}{.3mm} \\ Calificación:\rule{20mm}{.3mm} /100\end{bf}}
\footer{\begin{bf} Otoño -- 2015\end{bf}} {} {\begin{bf}Puntos: \rule{31.65mm}{.3mm}\end{bf}}
\runningheader{\begin{bf}Fonética y español oral\end{bf}}{}{\begin{bf}Prueba I\end{bf}}
\addpoints


\begin{document}

\vspace{0.1in} 

\section{Escribe las respuestas correctas: \_\_\_\_\_\_\_\_/55}
\noindent \begin{bf}A. (30 puntos)\end{bf}

\renewcommand{\arraystretch}{1.63}
	\begin{tabular}{@{}lcp{1in}@{}}
	 1.  & \_\_\_\_\_\_\_\_\_\_\_\_\_\_\_\_\_\_ & Aporta dos ejemplos de grafemas.                           \\
	 2.  & \_\_\_\_\_\_\_\_\_\_\_\_\_\_\_\_\_\_ & Da un ejemplo de un dígrafo y su fonema correspondiente.   \\
	 3.  & \_\_\_\_\_\_\_\_\_\_\_\_\_\_\_\_\_\_ & ¿Cuál es la ciencia que estudia las lenguas humanas?       \\
	 4.  & \_\_\_\_\_\_\_\_\_\_\_\_\_\_\_\_\_\_ & Transcribe la palabra ``hacha''                                                                  \\
	 5.  & \_\_\_\_\_\_\_\_\_\_\_\_\_\_\_\_\_\_ & Escribe un ejemplo de un fonema.                                \\
	 6.  & \_\_\_\_\_\_\_\_\_\_\_\_\_\_\_\_\_\_ & ¿Cuántos fonemas se corresponden con el grafema ``x''?          \\
	 7.  & \_\_\_\_\_\_\_\_\_\_\_\_\_\_\_\_\_\_ & Da un ejemplo de un núcleo vocálico en una sílaba y subráyalo                              \\
	 8.  & \_\_\_\_\_\_\_\_\_\_\_\_\_\_\_\_\_\_ & ¿Cuántos grafemas le corresponde al fonema /s/? Cuáles son? \\
	 9.  & \_\_\_\_\_\_\_\_\_\_\_\_\_\_\_\_\_\_ & ¿Qué tipo de diptongo se forma con una vocal cerrada/débil \\ & & y una vocal abierta/fuerte (en ese orden)?                           \\
	 10. & \_\_\_\_\_\_\_\_\_\_\_\_\_\_\_\_\_\_ & Cuando dos vocales contiguas están en la misma sílaba se dice \\ & & que hay un...         \\
	 11. & \_\_\_\_\_\_\_\_\_\_\_\_\_\_\_\_\_\_ & Aporta un ejemplo de un grafema que no tiene fonema.            \\
	 12. & \_\_\_\_\_\_\_\_\_\_\_\_\_\_\_\_\_\_ & Silabifica la secuencia de palabras ``el otro''.        \\
	 13. & \_\_\_\_\_\_\_\_\_\_\_\_\_\_\_\_\_\_ & Da un ejemplo de una vocal fuerte.                     \\
	 14. & \_\_\_\_\_\_\_\_\_\_\_\_\_\_\_\_\_\_ & Provee un ejemplo de una vocal débil.                  \\
	 15. & \_\_\_\_\_\_\_\_\_\_\_\_\_\_\_\_\_\_ & Da un ejemplo de una palabra que contenga un hiato y subráyalo.     \\
\end{tabular}
		
\vspace{3mm}\noindent \begin{bf}B. (25 puntos)\end{bf}

		\begin{questions}
			\question ¿Cuál es la diferencia entre una vocal débil y una fuerte? Provee ejemplos.
			\fillwithlines{1.45in}
			\question Explica qué es una correspondencia exclusiva y da un ejemplo.
			\fillwithlines{1.75in}
			\question ¿Cómo se caracterizan las vocales del español con respecto a las del inglés?
			\fillwithlines{1.75in}
			\question ¿Qué es una schwa? ¿Cuál es el símbolo fonética que representa este sonido? ¿Qué retos presenta para los angloparlantes?
			\fillwithlines{1.75in}
			\question ¿Cuál es la diferencia entre un diptongo acreciente, creciente y un decreciente? Da ejemplos:
			\fillwithlines{1.75in}
		\end{questions}
		
\section{Haz un tic (\checked) al lado de las palabras que contengan un diptongo y una cruz (X) al lado de las palabras que contengan un hiato (algunas de ellas no tienen ninguno de los dos): \_\_\_\_\_\_\_\_ /15 puntos}

\vspace{5mm}
\hspace{1mm}\begin{bf}MODELO: destrucción \hspace{5mm}(\checked) \hspace{9mm}veo \hspace{14mm}( X ) \hspace{7.5mm}regalo \hspace{12.25mm}(\hspace{2.5mm})\end{bf}\\
\renewcommand{\arraystretch}{1.275}
\vspace{1.5mm}

	\begin{tabular}{lclclc}
		\hspace{20mm} iguana		&	\hspace{5mm}(\hspace{2mm})	\hspace{5mm} & realidad	    & \hspace{5mm}(\hspace{2mm}) \hspace{5mm} & dirección	& \hspace{5mm}(\hspace{2mm})\\
		\hspace{20mm} yegua			&	\hspace{5mm}(\hspace{2mm})	\hspace{5mm} & yogur		& \hspace{5mm}(\hspace{2mm}) \hspace{5mm} & callado		& \hspace{5mm}(\hspace{2mm})\\
		\hspace{20mm} inutilidad	&	\hspace{5mm}(\hspace{2mm})	\hspace{5mm} & hallar	    & \hspace{5mm}(\hspace{2mm}) \hspace{5mm} & ruido		& \hspace{5mm}(\hspace{2mm})\\
		\hspace{20mm} distraer		&	\hspace{5mm}(\hspace{2mm})	\hspace{5mm} & caimán	    & \hspace{5mm}(\hspace{2mm}) \hspace{5mm} & ¡ay!		& \hspace{5mm}(\hspace{2mm})\\
		\hspace{20mm} constancia	&	\hspace{5mm}(\hspace{2mm})	\hspace{5mm} & hueco		& \hspace{5mm}(\hspace{2mm}) \hspace{5mm} & nuestro		& \hspace{5mm}(\hspace{2mm})\\
	\end{tabular}

\section{Menciona si los siguientes fonemas tienen correspondencia exclusiva. Explica por qué y menciona los grafemas correspondientes: \_\_\_\_\_\_\_\_ /10 puntos}

	\begin{questions}
		\question /t/ 
		\fillwithlines{.75in}
		\question /s/
		\fillwithlines{.75in}
		\question /k/
		\fillwithlines{.75in}
		\question /x/
		\fillwithlines{.75in}
		\question /u/
		\fillwithlines{.75in}
	\end{questions}
	
\section{Escribe una trascripción fonémica (legible) de las siguientes palabras. No te olvides de incluir los diptongos: \_\_\_\_\_\_\_\_/20 puntos}


\renewcommand{\arraystretch}{1.75}
	\begin{tabular}{llc}
		1.  & Genocidio		& \hspace{5mm}\rule{55mm}{.3mm} \\
		2.  & Bendición		& \hspace{5mm}\rule{55mm}{.3mm} \\
		3.  & Vibración		& \hspace{5mm}\rule{55mm}{.3mm} \\
		4.  & Cacería		& \hspace{5mm}\rule{55mm}{.3mm} \\
		5.  & Guitarra		& \hspace{5mm}\rule{55mm}{.3mm} \\
		6.  & Enredado		& \hspace{5mm}\rule{55mm}{.3mm} \\
		7.  & Recital		& \hspace{5mm}\rule{55mm}{.3mm} \\
		8.  & Zoológico		& \hspace{5mm}\rule{55mm}{.3mm} \\
		9.  & Quitamanchas	& \hspace{5mm}\rule{55mm}{.3mm} \\
		10. & Chorizo		& \hspace{5mm}\rule{55mm}{.3mm} \\
	\end{tabular}
	
\section{Puntos adicionales: \_\_\_\_\_\_\_\_/5 puntos}	

	\begin{questions}
		\question[2] Su amigo americano quiere aprender a pronunciar bien el español. ¿Qué consejos le darías?
		\fillwithlines{1in}
		\question[3] Diga si las palabras a continuación son agudas, llanas, esdrújulas o sobresdrújulas y tilde la palabra si hace falta. Explique \underline{por qué} es necesario o no.
	\end{questions}

\noindent Modelo:

	\begin{tabular}{@{}lcllp{3in}}
	Movil & $\rightarrow$ & \textipa{/\textprimstress mo.bil/} & \underline{Llana} & Sí, porque es una palabra llana y termina en `l'. \\
	& & & & \\ [-2ex]
	Tenia     & $\rightarrow$ & \textipa{/te.\textprimstress ni.a/} & \_\_\_\_\_\_\_\_\_\_\_ & \_\_\_\_\_\_\_\_\_\_\_\_\_\_\_\_\_\_\_\_\_\_\_\_\_\_\_\_\_\_\_\_\_\_\_\_\_\_\\
	Cafe      & $\rightarrow$ & \textipa{/ka.\textprimstress fe/} & \_\_\_\_\_\_\_\_\_\_\_ & \_\_\_\_\_\_\_\_\_\_\_\_\_\_\_\_\_\_\_\_\_\_\_\_\_\_\_\_\_\_\_\_\_\_\_\_\_\_\\
	Vereis    & $\rightarrow$ & \textipa{/be.\textprimstress Re\textsubarch{i}s/} & \_\_\_\_\_\_\_\_\_\_\_ & \_\_\_\_\_\_\_\_\_\_\_\_\_\_\_\_\_\_\_\_\_\_\_\_\_\_\_\_\_\_\_\_\_\_\_\_\_\_\\
	\end{tabular}


\end{document}