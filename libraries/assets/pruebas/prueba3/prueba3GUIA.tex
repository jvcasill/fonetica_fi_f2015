%
%  Created by Joseph Casillas on 2011-09-15.
%  Copyright (c) 2011 . All rights reserved.
%
%
\documentclass[12pt]{exam}
\usepackage{listings}
\usepackage{pdfsync}
\usepackage{ling}
\usepackage{multicol}
\usepackage{wasysym}
\usepackage{booktabs}

\setlength{\topmargin}{-0.1in}
\usepackage{sectsty}
\sectionfont{\normalsize}


\firstpageheader{\begin{bf} Fonética y español oral -- Casillas\\ Prueba III\end{bf}} {} {\begin{bf}Alumno: \rule{33.65mm}{.3mm} \\ Calificación:\rule{20mm}{.3mm} /100\end{bf}}
\footer{\begin{bf} Otoño -- 2015\end{bf}} {} {\begin{bf}Puntos: \rule{31.65mm}{.3mm}\end{bf}}
\runningheader{\begin{bf}Fonética y español oral\end{bf}}{}{\begin{bf}Prueba III\end{bf}}
\addpoints


\begin{document}

\vspace{0.1in}

\section{Escriba el alófono que corresponde a la descripción. (10 puntos): \_\_\_\_\_\_\_\_/10}
\renewcommand{\arraystretch}{1.75}
	\begin{tabular}{@{}lll@{}}
	1.  & \_\_\_\_\_\_\_\_\_\_\_\_\_\_\_\_\_\_\_\_\_\_\_ & fricativo bilabial sonoro    \\
	\end{tabular}

\vspace{.2in}

\section{Describa los siguientes alófonos (30 puntos): \_\_\_\_\_\_\_\_/30}

	\begin{center}
		\begin{tabular}{|c|c|c|c|}
			\hline
			    & Modo de articulación & Punto de articulación & Sonoridad (sordo/sonoro)\\
			\hline
			[\texttheta] & & & \\
			\hline
		\end{tabular}
	\end{center}

\vspace{.2in}

\section{Escriba la respuesta correcta en el espacio a la izquierda de la oración. (30 puntos): \_\_\_\_\_\_\_\_/30}

\renewcommand{\arraystretch}{1.75}
	\begin{tabular}{@{}llp{2cm}@{}}
	 1.  & \_\_\_\_\_\_\_\_\_\_\_\_\_\_\_\_\_\_ & Un sonido que se produce sin vibración de las cuerdas vocales se \\&& denomina        \\
	\end{tabular}

\vspace{.2in}

\section{Conteste las siguientes preguntas (10 puntos): \_\_\_\_\_\_\_\_/10}
\begin{itemize}
	\item Respuestas cortas (2-4 oraciones)
	\item Valen 3, 4 y 3 puntos.
\end{itemize}

\vspace{.2in}

\section{Escriba una transcripción /fonológica/ y [fonética] de las siguientes oraciones. Incluya los fonemas (2) o alófonos (si es necesario), los encadenamientos (.5) y separe en sílabas (1). Elija la variedad (10 puntos): \_\_\_\_\_\_\_\_/10 }

\vspace{.2in}

\section{Dé una transcripción [fonética] de las siguientes palabras. Variedad peninsular estándar (10 puntos): \_\_\_\_\_\_\_\_/10 } % (fold)

	\renewcommand{\arraystretch}{1.5}
	\begin{tabular}{lllllr}
		1.  & gitano        & & [ &  \phantom{hello world, hello world} & ] \\
	\end{tabular}



\end{document}