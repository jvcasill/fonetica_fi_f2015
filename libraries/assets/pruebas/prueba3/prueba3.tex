%
%  Created by Joseph Casillas on 2011-09-15.
%  Copyright (c) 2011 . All rights reserved.
%
%
\documentclass[12pt]{exam}
\usepackage{listings}
\usepackage{pdfsync}
\usepackage{ling}
\usepackage{multicol}
\usepackage{wasysym}
\usepackage{booktabs}

\setlength{\topmargin}{-0.1in}
\usepackage{sectsty}
\sectionfont{\normalsize}


\firstpageheader{\begin{bf} Fonética y español oral -- Casillas\\ Prueba III\end{bf}} {} {\begin{bf}Alumno: \rule{33.65mm}{.3mm} \\ Calificación:\rule{20mm}{.3mm} /100\end{bf}}
\footer{\begin{bf} Otoño -- 2015\end{bf}} {} {\begin{bf}Puntos: \rule{31.65mm}{.3mm}\end{bf}}
\runningheader{\begin{bf}Fonética y español oral\end{bf}}{}{\begin{bf}Prueba III\end{bf}}
\addpoints


\begin{document}

\vspace{0.1in}

\section{Escriba el alófono que corresponde a la descripción. (10 puntos): \_\_\_\_\_\_\_\_/10}
\renewcommand{\arraystretch}{1.75}
	\begin{tabular}{@{}lll@{}}
	1.  & \_\_\_\_\_\_\_\_\_\_\_\_\_\_\_\_\_\_\_\_\_\_\_ & fricativo bilabial sonoro    \\
	2.  & \_\_\_\_\_\_\_\_\_\_\_\_\_\_\_\_\_\_\_\_\_\_\_ & fricativo dental sonoro      \\
	3.  & \_\_\_\_\_\_\_\_\_\_\_\_\_\_\_\_\_\_\_\_\_\_\_ & fricativo velar sonoro       \\
	4.  & \_\_\_\_\_\_\_\_\_\_\_\_\_\_\_\_\_\_\_\_\_\_\_ & nasal velar sonoro           \\
	5.  & \_\_\_\_\_\_\_\_\_\_\_\_\_\_\_\_\_\_\_\_\_\_\_ & africado alveopalatal sonoro \\
	6.  & \_\_\_\_\_\_\_\_\_\_\_\_\_\_\_\_\_\_\_\_\_\_\_ & lateral alveopalatal sonoro  \\
	7.  & \_\_\_\_\_\_\_\_\_\_\_\_\_\_\_\_\_\_\_\_\_\_\_ & fricativo alveolar sonoro    \\
	8.  & \_\_\_\_\_\_\_\_\_\_\_\_\_\_\_\_\_\_\_\_\_\_\_ & lateral palatal sonoro       \\
	9.  & \_\_\_\_\_\_\_\_\_\_\_\_\_\_\_\_\_\_\_\_\_\_\_ & fricativo alveolar sordo     \\
	10.	& \_\_\_\_\_\_\_\_\_\_\_\_\_\_\_\_\_\_\_\_\_\_\_ & vibrante simple sonoro       \\
	\end{tabular}

\section{Describa los siguientes alófonos (30 puntos): \_\_\_\_\_\_\_\_/30}

	\begin{center}
		\begin{tabular}{|c|c|c|c|}
			\hline
			    & Modo de articulación & Punto de articulación & Sonoridad (sordo/sonoro)\\
			\hline
			[\textgamma] & & & \\
			\hline
			[x] & & & \\
			\hline
			[r] & & & \\
			\hline
			[n] & & & \\
			\hline
			[\textsubbridge{n}] & & & \\
			\hline
			[\dh] & & & \\
			\hline
			[\textltailm] & & & \\
			\hline
			[\textbeta] & & & \\
			\hline
			[z] & & & \\
			\hline
			[\textipa{J}] & & & \\
			\hline
		\end{tabular}
	\end{center}



\section{Escriba la respuesta correcta en el espacio a la izquierda de la oración. (30 puntos): \_\_\_\_\_\_\_\_/30}

\renewcommand{\arraystretch}{1.75}
	\begin{tabular}{@{}llp{2cm}@{}}
	 1.  & \_\_\_\_\_\_\_\_\_\_\_\_\_\_\_\_\_\_ & Un sonido que se produce sin vibración de las cuerdas vocales se \\&& denomina        \\
	 2.  & \_\_\_\_\_\_\_\_\_\_\_\_\_\_\_\_\_\_ & Un ejemplo de un articulador activo es                                           \\
	 3.  & \_\_\_\_\_\_\_\_\_\_\_\_\_\_\_\_\_\_ & Un fonema producido con algún tipo de obstrucción del paso del aire \\&& es una       \\
	 4.  & \_\_\_\_\_\_\_\_\_\_\_\_\_\_\_\_\_\_ & Un ejemplo de articulador pasivo es \\
	 5.  & \_\_\_\_\_\_\_\_\_\_\_\_\_\_\_\_\_\_ & ¿Qué tienen en común las siguientes consonantes /k x s f/? \\&& Todas son \_\_\_\_\_\_.\\
	 6.  & \_\_\_\_\_\_\_\_\_\_\_\_\_\_\_\_\_\_ & El modo de articulación en el cual se produce un obstáculo total y \\&& luego un obstáculo parcial se llama \_\_\_\_\_\_.\\
	 7. & \_\_\_\_\_\_\_\_\_\_\_\_\_\_\_\_\_\_ & Las vocales altas son \_\_\_\_\_\_ y \_\_\_\_\_\_.\\
	 8. & \_\_\_\_\_\_\_\_\_\_\_\_\_\_\_\_\_\_ & ¿Qué tiene en común los siguientes fonemas: /s, r, l, n/?\\
	 9. & \_\_\_\_\_\_\_\_\_\_\_\_\_\_\_\_\_\_ & Cuándo dos vocales contiguas están en dos sílabas diferentes se dice \\&& que hay un \_\_\_\_\_\_.\\
	 10. & \_\_\_\_\_\_\_\_\_\_\_\_\_\_\_\_\_\_ & ¿Cuál es el elemento obligatorio de una sílaba?\\
	 11. & \_\_\_\_\_\_\_\_\_\_\_\_\_\_\_\_\_\_ & ¿Cuál es la estructura silábica preferida en español?\\
	 12. & \_\_\_\_\_\_\_\_\_\_\_\_\_\_\_\_\_\_ & Aporte un ejemplo de una sílaba que contenga coda y \underline{subráyela} \\
	 13. & \_\_\_\_\_\_\_\_\_\_\_\_\_\_\_\_\_\_ & Aporte un ejemplo de un grupo consonántico prohibido en español. \\
	 14. & \_\_\_\_\_\_\_\_\_\_\_\_\_\_\_\_\_\_ & ¿Qué tienen en común los siguientes fonemas consonánticos: /j, ñ/? \\
	 15. & \_\_\_\_\_\_\_\_\_\_\_\_\_\_\_\_\_\_ & Aporte un ejemplo de una palabra que contenga una sílaba abierta y \\&& \underline{subraye} la sílaba\\
	\end{tabular}

\section{Conteste las siguientes preguntas (10 puntos): \_\_\_\_\_\_\_\_/10}
	\begin{questions}
		\question ¿Cuáles son los tres procesos fonológicos que hemos visto en clase? ¿Qué tienen en común? ¿En qué se diferencian? Dé ejemplos.
		\fillwithlines{2in}
		\question Describa el proceso de asimilación que ocurre con las nasales en español. ¿Cuántos alófonos hay? ¿Cuáles son?
		\fillwithlines{2in}
		\question Describa el proceso de asimilación que ocurre con la /s/ en español. ¿Cómo varía con respecto al inglés?
		\fillwithlines{2in}
	\end{questions}


\section{Escriba una transcripción /fonológica/ y [fonética] de las siguientes oraciones. Incluya los fonemas (2) o alófonos (si es necesario), los encadenamientos (.5) y separe en sílabas (1) (10 puntos): \_\_\_\_\_\_\_\_/10 }

	\begin{questions}
		\question Raquel estuvo esperando todo el día en frente del mismo banco. 
		\fillwithlines{.5in}
		\question Al oír esos gritos uno se da cuenta de que el corazón de toda la hinchada ha entrado en el juego. 
		\fillwithlines{.5in}
	\end{questions}

\section{Dé una transcripción [fonética] de las siguientes palabras (10 puntos): \_\_\_\_\_\_\_\_/10 } % (fold)

	\renewcommand{\arraystretch}{1.5}
	\begin{tabular}{lllllr}
		1.  & mismo        & & [ &  \phantom{hello world, hello world} & ] \\
		2.  & dedo         & & [ & & ] \\
		3.  & manco        & & [ & & ] \\
		4.  & en Finlandia & & [ & & ] \\
		5.  & los dedos    & & [ & & ] \\
		6.  & algas        & & [ & & ] \\
		7.  & ando         & & [ & & ] \\
		8.  & ancho        & & [ & & ] \\
		9.  & saldo        & & [ & & ] \\
		10. & un yate      & & [ & & ] \\
	\end{tabular}


\section{Puntos extras (3 como máximo): \_\_\_\_\_\_\_\_/3 }
\begin{questions}
	\question ¿Cuál es el dialecto más prestigioso del español? ¿En que se basa su respuesta?
	\fillwithlines{1.5in}
\end{questions}






	
\end{document}