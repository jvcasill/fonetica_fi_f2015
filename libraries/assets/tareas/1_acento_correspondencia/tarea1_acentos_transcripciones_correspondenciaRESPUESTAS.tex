\documentclass{article}
\usepackage{ling}
\usepackage[hmargin=2.54cm,vmargin=2.54cm]{geometry}
\geometry{a4paper}
\usepackage{multirow}
\usepackage{fancyhdr} 
\pagestyle{fancy} 
\renewcommand{\headrulewidth}{0pt}
\lhead{\textbf{La transcripción y la acentuación}}\chead{}\rhead{\textbf{Nombre:\_\_\_\_\_\_\_\_\_\_\_\_\_\_\_}}
\lfoot{Español y comunicación oral -- Casillas}\cfoot{\thepage}\rfoot{otoño -- 2015}
\usepackage{setspace}
\setstretch{1.667}


\begin{document}

\noindent \textbf{Transcriba las siguientes palabras. Indentifique los diptongos (crecientes/decrecientes) y los hiatos:}

\vspace{.2in}
\begin{tabular}{@{}llr@{}}

\hline
Palabra         & Transcripción fonémica                                   & Diptongo/hiato       \\
\hline
``lingüístico'' & \textipa{/lin.g\textsubarch{u}is.\textprimstress ti.ko/} & diptongo acreciente  \\
``cuota''       & \textipa{/\textprimstress k\textsubarch{u}o.ta/}         & diptongo creciente   \\
``reina''       & \textipa{/\textprimstress re\textsubarch{i}.na/}         & diptongo decreciente \\
``soy''         & \textipa{/so\textsubarch{i}/}                            & diptongo decreciente \\
``cual''        & \textipa{/k\textsubarch{u}al/}                           & diptongo creciente   \\
``aislar''      & \textipa{/a\textsubarch{i}s.\textprimstress laR/}        & diptongo decreciente \\
``mientras''    & \textipa{/\textprimstress m\textsubarch{i}en.tRas/}      & diptongo creciente   \\
``neutro''      & \textipa{/\textprimstress ne\textsubarch{u}.tRo/}        & diptongo decreciente \\
``Miriam''      & \textipa{/\textprimstress mi.R\textsubarch{i}am/}        & diptongo creciente   \\
``ahora''       & \textipa{/a.\textprimstress o.Ra/}                       & hiato                \\
\hline
\end{tabular}

\vspace{.2in}





\noindent \textbf{Grafemas y fonemas con correspondencia exclusiva:}

\vspace{.1in}
\begin{center}
\begin{tabular}{@{}ccp{5in}@{}}
\hline \\ [-3ex]
Grafema & Fonema         & Ejemplos \\ [2ex]
\hline \\ [-3ex]
``d''   & /d/            & ``dedo'' - \textipa{/\textprimstress de.do/} \\ [2ex]
``p''   & /p/            & ``papa'' - \textipa{/\textprimstress pa.pa/} \\ [2ex]
``l''   & /l/            & ``lana'' - \textipa{/\textprimstress la.na/} \\ [2ex]
``ch''  & /\textteshlig/ & ``chepa'' - \textipa{/\textprimstress\textteshlig e.pa/} \\ [2ex]
``t''   & /t/            & ``tela'' - \textipa{/\textprimstress te.la/} \\ [2ex]
``f''   & /f/            & ``fila'' - \textipa{/\textprimstress fi.la/} \\ [2ex]
``a''   & /a/            & ``ala'' - \textipa{/\textprimstress a.la/} \\ [2ex]
``o''   & /o/            & ``ola'' - \textipa{/\textprimstress o.la/} \\ [2ex]
``e''   & /e/            & ``era'' - \textipa{/\textprimstress e.Ra/} \\ [2ex]
\hline
\end{tabular}
\end{center}

\pagebreak







\noindent \textbf{Grafemas y fonemas sin correspondencia exclusiva:}

\vspace{-.1in}
\begin{center}
\begin{tabular}{@{}ccp{4.75in}@{}}
\hline \\ [-3ex]
Grafema                & Fonema                                & Ejemplos \\ [2ex]
\hline \\ [-3ex]
``b''                  & \multirow{2}{*}{\textipa{/b/}}        & ``baso'' - \textipa{/\textprimstress ba.so/} \\ [.25ex]
``v''                  &                                       & ``vaso'' - \textipa{/\textprimstress ba.so/} \\ [.25ex] \hline
``c''                  & \multirow{3}{*}{\textipa{/s/}}        & ``celo'' - \textipa{/\textprimstress se.lo/} \\ [.25ex]
``z''                  &                                       & ``zeta'' - \textipa{/\textprimstress se.ta/} \\ [.25ex]
``s''                  & (excluye \textipa{/\texttheta/})      & ``seta'' - \textipa{/\textprimstress se.ta/} \\ [.25ex] \hline
``c''                  & \multirow{3}{*}{\textipa{/k/}}        & ``cubo''  - \textipa{/\textprimstress ku.bo/} \\ [.25ex]
``qu''                 &                                       & ``queso'' - \textipa{/\textprimstress ke.so/} \\ [.25ex]
``k''                  &                                       & ``kilo''  - \textipa{/\textprimstress ki.lo/} \\ [.25ex] \hline
``g''                  & \textipa{/g/}                         & ``gato'' - \textipa{/\textprimstress ga.to/} \\ [.25ex] \hline
``g''                  & \multirow{3}{*}{\textipa{/x/}}        & ``gente'' - \textipa{/\textprimstress xen.te/} \\ [.25ex]
``j''                  &                                       & ``jeta'' - \textipa{/\textprimstress xe.ta/} \\ [.25ex]
``x''                  &                                       & ``M\'exico'' - \textipa{/\textprimstress me.xi.ko/} \\ [.25ex] \hline
``h''                  & Ø                                     & ``hola'' - \textipa{/\textprimstress o.la/} \\ [.25ex] \hline
``i''                  & \textipa{/i/}                         & ``pila'' - \textipa{/\textprimstress pi.la/} \\ [.25ex]
``y''                  & \textipa{/\textsubarch{i}/}           & ``hay'' - \textipa{/a\textsubarch{i}/} \\ [.25ex] \hline
``ll''                 & \multirow{2}{*}{\textipa{/\textctj/}} & ``llamar'' - \textipa{/\textctj a.\textprimstress maR/} \\ [.25ex]
``y''                  &                                       & ``yeso'' - \textipa{/\textprimstress \textctj e.so/} \\ [.25ex] \hline
\multirow{2}{*}{``m''} & \textipa{/m/}                         & ``album'' - \textipa{/\textprimstress al.bum/} \\ [.25ex]
                       & \textipa{/n/}                         & ``album'' - \textipa{/\textprimstress al.bun/} \\ [.25ex] \hline
\multirow{2}{*}{``n''} & \textipa{/m/}                         & ``'' - \textipa{//} \\ [.25ex]
                       & \textipa{/n/}                         & ``'' - \textipa{//} \\ [.25ex] \hline
\multirow{2}{*}{``r''} & \textipa{/r/}                         & ``rico'' - \textipa{/\textprimstress ri.ko/} \\ [.25ex]
                       & \textipa{/R/}                         & ``para'' - \textipa{/\textprimstress pa.Ra/} \\ [.25ex] \hline
``rr''                 & \multirow{2}{*}{\textipa{/r/}}        & ``parra'' - \textipa{/\textprimstress pa.ra/} \\ [.25ex]
``r''                  &                                       & ``rico'' - \textipa{/\textprimstress ri.ko/} \\ [.25ex] \hline
\multirow{3}{*}{``u''} & \textipa{/u/}                         & ``uso'' - \textipa{/\textprimstress u.so/} \\ [.25ex]
                       & \textipa{/\textsubarch{u}/}           & ``pausa'' - \textipa{/\textprimstress pa\textsubarch{u}.sa/} \\ [.25ex]
                       & \textipa{/w/}                         & ``hueso'' - \textipa{/\textprimstress we.so/} \\ [.25ex]
\hline
\end{tabular}
\end{center}






\pagebreak

\noindent \textbf{Coloque el acento en las palabras que lo precisen:}
\begin{enumerate}
	\item C\'esar Gonz\'alez record\'o en una transmisi\'on televisiva los duros años de detenci\'on y c\'arcel.
	\item El descubridor franc\'es preconiz\'o una combinaci\'on de tratamiento para combatir el virus.
	\item En los \'ultimos meses el d\'eficit p\'ublico alcanz\'o cuotas nunca vistas.
	\item La eficacia de esta terapia contra el Sida no est\'a probada.
	\item Despu\'es de cenar, Ram\'on se sent\'o en el sill\'on y ley\'o un rato.
	\item El joven explorador fue mordido por un r\'eptil y muri\'o en pocas horas.
	\item La comunicaci\'on y el di\'alogo son las premisas de toda relaci\'on duradera.
	\item Para m\'i el pan integral es mucho m\'as sabroso que el de harina blanca, ¿y para ti?
	\item S\'e muy bien que en aquella ocasi\'on tu hablaste mal de todos nosotros.
	\item M\'as vale que esto no lo digas en su presencia.
\end{enumerate}

\noindent \textbf{Diga si las palabras a continuación son agudas, llanas, esdrújulas o sobresdrújulas y tilde la palabra si hace falta. Explique \underline{por qué} es necesario o no:}\\

\noindent Modelo:\\

	\begin{tabular}{@{}lcllp{3.25in}}
	Movil & $\rightarrow$ & \underline{Móvil} & \underline{Llana} & Sí, lleva tilde porque es una palabra llana y termina en l. \\
	& & & & \\
	Casa      & $\rightarrow$ & Casa        & Llana            & No, es llana y termina en vocal. \\
	Esdrujula & $\rightarrow$ & Esdr\'ujula & Esdr\'ujula      & S\'i, es esdr\'ujula. \\
	Oir       & $\rightarrow$ & O\'ir       & Aguda            & S\'i, antidiptongo. \\
	Tenia     & $\rightarrow$ & Ten\'ia     & Llana            & S\'i, antidiptongo. \\
	Dandoselo & $\rightarrow$ & D\'andoselo & Sobreesdr\'ujula & S\'i, sobreesdr\'ujula. \\
	Dia       & $\rightarrow$ & D\'ia       & Llana            & S\'i, antidiptongo. \\
	Cafe      & $\rightarrow$ & Caf\'e      & Aguda            & S\'i, es aguda y termina en vocal. \\
	Vereis    & $\rightarrow$ & Ver\'eis    & Aguda            & S\'i, es aguda y termina en `s'. \\
	Ibamos    & $\rightarrow$ & \'Ibamos    & Esdr\'ujula      & S\'i, esdr\'jula. \\
	Guerra    & $\rightarrow$ & Guerra      & Llana            & No, es llana y termina en vocal. \\

	\end{tabular}






\end{document}