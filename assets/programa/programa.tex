%Mi primera plantilla
%Se ha diseñado con el objeto de usarse para clase

\documentclass[12pt]{article}
\usepackage{ling} 
\renewcommand{\rmdefault}{ptm}
\usepackage{setspace}
\setstretch{1.667}
\usepackage{colortbl}

%%% PAGE DIMENSIONS
\usepackage[hmargin=2.54cm,vmargin=2.54cm]{geometry}
\geometry{a4paper}% \geometry{landscape} % set up the page for landscape

%%% PACKAGES
\usepackage{booktabs} % for much better looking tables
\usepackage{array} % for better arrays (eg matrices) in maths
\usepackage{paralist} % very flexible & customisable lists (eg. enumerate/itemize, etc.)
\usepackage{verbatim} % adds environment for commenting out blocks of text & for better verbatim
\usepackage{subfig} % make it possible to include more than one captioned figure/table in a single float
% These packages are all incorporated in the memoir class to one degree or another...

%%% HEADERS & FOOTERS
\usepackage{fancyhdr} % This should be set AFTER setting up the page geometry
\pagestyle{fancy} % options: empty , plain , fancy
\renewcommand{\headrulewidth}{0pt} % customise the layout...
\lhead{}\chead{}\rhead{}
\lfoot{}\cfoot{\thepage}\rfoot{}


\begin{document}

\begin{center} \begin{bf} \Large{Fonética y español oral}\end{bf} \end{center}

\begin{singlespace}
\noindent Otoño 2015\\
\noindent Martes y jueves 10:30 -- 12:00 am \\
\noindent Lugar: 
\end{singlespace}

\begin{singlespace}
\noindent Instructor: Joseph Casillas\\
\noindent Office location: ML 209\\
\noindent Office phone: \\
\noindent Office hours: Every day after class \& by appointment\\
\noindent Email: jvcasill@email.arizona.edu\\
\end{singlespace}



\section{Materiales}
\subsection{Libro de texto}

\begin{singlespace}
	\begin{itemize}
	\itemsep=-2pt
		\item Schwegler, Armin \& Kempff, Juergen (eds.) 2007. Fonética y fonología españolas.  New York: Wiley. [3rd edition]
		\item Resúmenes de las presentaciones de clase:  D2l, Content
	\end{itemize}
\end{singlespace}

\subsection{Materiales de consulta recomendados}
\begin{singlespace}
	\begin{itemize}
	\itemsep=-2pt
		\item Barrutia, Richard \& Armin Schwegler (eds.) 1994. Fonética y fonología españolas: teoría y práctica.  New York: Wiley. [2nd edition] 
		\item Teschner, Richard V. 1996. Camino oral: fonética, fonología y práctica de los sonidos del español.  New York: McGraw-Hill. [Libro de texto y cassettes]. 
		\item Hammond, Robert. 2001. The Sounds of Spanish. Somerville, MA: Cascadilla. Audio files: http://www.cascadilla.com/ssaa/index.html.
		\item Piñeros, Carlos Eduardo. 2008. Estructura de los sonidos del español. Upper Saddle River, NJ: Pearson, Prentice Hall
	\end{itemize}
\end{singlespace}

\section{Objetivos del curso}
\begin{singlespace}
\noindent El objetivo primordial de este curso consiste en ayudar al hablante nativo de inglés a mejorar su pronunciación del español. Se intentará cumplir el mencionado objetivo creando conciencia en el estudiante de cómo funciona el sistema fonológico del español a varios niveles: a través de un mejor entendimiento de los conceptos fonéticos y fonológicos relevantes, de la aplicación de dichos conceptos mediante ejercicios de transcripción escrita y producción oral y del reconocimiento de rasgos fonéticos y fonológicos en el español hablado, todo ello sin olvidar el importante papel desempeñado por la variación dialectal. El curso proporciona al estudiante el ambiente relajado y de apoyo que es necesario para reducir al máximo las inhibiciones y obstáculos que a menudo se interponen a la práctica de la pronunciación.
\end{singlespace}

\section{Asistencia a clase}
\subsection{}
\begin{singlespace}
\noindent Puesto que la mayoría de los ejercicios de transcripción y de producción/práctica oral se harán EN CLASE, es esencial para obtener una buena nota en el curso que el estudiante asista regularmente a clase y que participe activamente en la misma. Los conceptos y conocimientos impartidos en SPAN 340 no se prestan a ser adquiridos meramente através de la lectura (independientemente del número de horas que se le dediquen). El repaso y la práctica constantes y regulares son absolutamente necesarios. Cuando un estudiante falta a clase, no puede realizar la práctica requerida para la adquisición de la fonología de la lengua y por lo tanto no obtendrá una buena nota. Debido al tiempo reducido que tenemos durante esta sesión, no se permitirá ni una ausencia.
\end{singlespace}

\subsection{}
\begin{singlespace}
\noindent Las ausencias se considerarán justificadas sólo en caso de emergencia por causas de salud o de familia (el alumno debe tener disponible la documentación necesaria para la verificación). [!OJO! Si faltas a una prueba, debes presentar documentación que demuestre que tal ausencia fue debida a enfermedad u otro tipo de emergencia seria. En caso contrario, la nota correspondiente a tal prueba será un cero.]
\end{singlespace}

\section{Calificaciones}
\vspace{.25in}

\begin{center}
	\begin{tabular*}{0.5\textwidth}{@{\extracolsep{\fill}}|lcccr|@{}}
		\hline
		\multicolumn{5}{|c|}{Distribución de las notas} \\
		\hline
		A & = & 90\% & --- & 100\%   \\
		B & = & 80\% & --- & 89\%   \\
		C & = & 70\% & --- & 79\%   \\
		D & = & 60\% & --- & 69\%   \\
		F & = & 0\%  & --- & 59\%   \\
		\hline
	\end{tabular*}

\vspace{.25in}

	\begin{tabular*}{0.5\textwidth}{@{\extracolsep{\fill}}|lr|@{}}
		\hline
		\multicolumn{2}{|c|}{Componentes de la nota final} \\
		\hline
		Tareas (incluyen pop quizes)         & 10\% \\
		Preparación, participación y actitud & 10\% \\
		Pruebas                              & 40\% \\
		Grabación (incluyendo cita)          & 10\% \\
		Presentación vídeo                   & 10\% \\
		Examen final (oral)                  & 20\% \\
		\hline
	\end{tabular*}
\end{center}



\section{Tareas}
\subsection{}
\begin{singlespace}
\noindent Cada estudiante deberá preocuparse por mantener el contacto (teléfono, email, etc.) con al menos otras 2 personas de la clase, ya que es responsabilidad del propio estudiante entregar a tiempo la tarea (ya sea mediante un compañero o en persona), y también enterarse de la tarea asignada para la siguiente clase en caso de ausencia. Recuerda que si faltas a clase, TÚ ERES EL ÚNICO RESPONSABLE de recuperar el trabajo. El profesor está dispuesto a ayudarte durante horas de oficina y a otras horas, siempre que su horario lo permita, pero la responsabilidad es TUYA. TÚ eres el que debe iniciar el proceso de pedir ayuda (ya sea la del profesor o la de tus compañeros) para ponerte al día.\\
\begin{bf}
\noindent La tareas asignadas deben ser corregidas por el estudiante usando las hojas de soluciones que vienen en el libro (también en D2l), con tinta de otro color y tomando nota de los errores que no se entiendan, para que el profesor pueda explicarlos en la misma tarea.
\end{bf}
\end{singlespace}

\subsection{}
\begin{singlespace}
\noindent No se aceptará ninguna tarea después del plazo asignado (a no ser que se haya obtenido una excusa por anticipado por motivos justificados). Una ausencia (aún con justificación) no sirve de excusa para entregar una tarea fuera de plazo a no ser que el motivo de la ausencia fuese tal que impidiese al alumno hacer la tarea y/o entregarla por medio de un compañero o compañera.
\end{singlespace}

\section{Preparación, participación y actitud}
\begin{singlespace}
\noindent Además de la participación oral (individual y en grupo), parte del trabajo de clase consistirá en ejercicios de práctica escritos. Algunos serán individuales y otros se realizarán en grupos. Todo esto será parte de la nota de Preparación, participación y actitud. Todos los alumnos deben estar preparados y saber bien los temas a tratar en cada clase, ya que la nota de participación dependerá de ello.
\end{singlespace}

\section{Pruebas}
\begin{singlespace}
\noindent Haremos cuatro pruebas (de unos 45 m) durante el semestre. Cada una de ellas cubrirá todo el material visto a partir de la última prueba, lo que no quiere decir que se vaya a excluir el contenido tratado en pruebas anteriores. Esto es absolutamente necesario, ya que los conceptos y conocimientos de este curso dependen unos de otros y son imposibles de tratar aisladamente. Además, es muy difícil pasar a temas nuevos sin dominar completamente los anteriores. No dejes que se te acumule el trabajo. Pasado un cierto punto, te resultará imposible ponerte al día. \\

\noindent NO se darán exámenes de recuperación (No make-ups). Si un estudiante falta a una prueba con una ausencia justificada, el porcentaje de la nota final correspondiente a la misma se distribuirá entre el resto de las pruebas.
\end{singlespace}

\section{Grabación}
\begin{singlespace}
\noindent Se asignará unas tareas de grabación durante el semestre. El estudiante tendrá que hacer una grabación de audio de la lectura de varios ejercicios y/o de una conversación. La grabación se entregará por medio del Dropbox de D2l. Cada estudiante deberá hacer al menos una cita con el profesor para obtener comentarios y sugerencias para mejorar su pronunciación en base a esta grabación.
\end{singlespace}

\section{Video y presentación}
\begin{singlespace}
\noindent Cada estudiante debe escoger un video corto (3-4 minutos) de anuncio o canción para niños disponible en Internet y preparar su presentación en clase. La presentación consistirá en lo siguiente: 
\begin{enumerate}
	\item Preparar el guión del diálogo/letra del video y un ejercicio de pronunciación basado en el vídeo.
	\item Estar listo para hacer comentarios/ contestar preguntas sobre el vídeo.
\end{enumerate}

Este trabajo puede ser individual o en pareja. El estudiante encargado de la presentación debe enviar al profesor el enlace del video seleccionado a más tardar dos días antes del día en que se va a presentar. El estudiante puede solicitar la ayuda del profesor con la preparación pero debe de hacerlo también a más tardar dos días antes de la presentación.
\end{singlespace}

\section{Examen final}
\begin{singlespace}
\noindent El examen final tendrá lugar el último día de la sesión. El objetivo es tener una forma de evaluación que permita ver cómo ha aplicado el alumno a su español los conceptos aprendidos en el curso. Se proveerán más detalles más adelante en el semestre.
\end{singlespace}

\section*{Code of Academic Integrity}
\begin{singlespace}
\noindent The instructor and the Program Director will initiate an academic integrity case against students suspected of cheating, plagiarizing, or aiding others in dishonest academic behavior. Students are responsible for reading and understanding the Code of Academic Integrity, please refer to: http://studpubs.web.arizona.edu/policies/cacaint.htm. Examples of academic dishonesty include, but are not limited to, plagiarism, cheating, and aiding and abetting dishonesty. If the instructor suspects that a Code of Academic Violation has occurred, she/he in accordance with the Program Director shall impose any one of the following or a combination of the following sanctions: (1) Loss of credit for work involved, (2) Reduction in grade for the entire course, (3) Failing grade, (4) Disciplinary probation. For policies against threatening behavior by students, please visit: http://policy.web.arizona.edu/-policy/threaten.shtml.
\end{singlespace}

\section*{Disability}
\begin{singlespace}
\noindent Those students who are registered with the Disability Resource Center must submit appropriate documentation to the instructor if they are requesting reasonable accommodations. Please refer to: http://drc.arizona.edu/instructor/svllabusstatement.shtml.
\end{singlespace}

\pagebreak

\section*{Temario}
\begin{singlespace}
\noindent Nota: el siguiente temario es provisional. El objetivo es proporcionar una visión de conjunto de los temas a tratar y de las fechas aproximadas. A medida que avancemos en el curso, iremos proporcionando detalles y fechas más exactas más adecuadas al ritmo de la clase.
\end{singlespace}

\vspace{.5in}
\begin{center}
\begin{tabular*}{\textwidth}{@{}|lrrcp{8.9cm}|@{}}
	\hline
	\rowcolor[gray]{0.9}
	Semana & Día  & Fecha & Capítulos & Temas \\
	\hline
	\rowcolor[gray]{0.8}
	Semana 1 & 1  & 19/12 &    3      & Intro al curso. La sílaba. Diptongos. Hiatos.     \\
	\rowcolor[gray]{0.9}
			 & 2  & 20/12 &    4      & Consonantes [\textctj] y [w]. Diptongos y semivocales.      \\
	\rowcolor[gray]{0.8}
			 & 3  & 21/12 &  5 y 6    & Sílabas, vocales y ritmo. Vocales en contacto. Schwa.      \\
	\rowcolor[gray]{0.9}
			 & 4  & 22/12 &    2*     & Fonética articulatoria: vocales.      \\
	\rowcolor[gray]{0.8}
			 & 5  & 23/12 &    7      & Fonética articulatoria: consonantes.      \\
	\rowcolor[gray]{0.9}
			 & -  & 24/12 &           & Vacaciones de navidad       \\
	\rowcolor[gray]{0.8}
			 & -  & 25/12 &           & Vacaciones de navidad       \\
	\rowcolor[gray]{0.9}
			 & -  & 26/12 &           & Vacaciones de navidad       \\
	\rowcolor[gray]{0.8}
			 & -  & 27/12 &           & Vacaciones de navidad       \\
	\rowcolor[gray]{0.9}
	Semana 2 & 6  & 28/12 &  9 y 8    & El fonema. Transcripción fonética.      \\
	\rowcolor[gray]{0.8}
			 & 7  & 29/12 &   10*     & Oclusivas sordas, africadas.      \\
	\rowcolor[gray]{0.9}
			 & 8  & 30/12 &   9       & Obstruyentes sonoras /b, d, g/ /\textctj/       \\
	\rowcolor[gray]{0.8}
			 & -  & 31/12 &           & Vacaciones de nochevieja      \\
	\rowcolor[gray]{0.9}
			 & -  & 1/1   &           & Vacaciones de nochevieja      \\
	\rowcolor[gray]{0.8}
			 & -  & 2/1   &           & Vacaciones de nochevieja      \\
	\rowcolor[gray]{0.9}
	Semana 3 & 9  & 3/1   &   15      & Fricativas: /f, s, x/      \\
	\rowcolor[gray]{0.8}
			 & 10 & 4/1   &  11 y 14  & Nasales y laterales      \\
	\rowcolor[gray]{0.9}
			 & 11 & 5/1   &  13 y 16  & Vibrantes. La entonación      \\
	\rowcolor[gray]{0.8}
			 & 12 & 6/1   &   19*     & El español penínsular y el español americano     \\
	\rowcolor[gray]{0.9}
	Semana 4 & 13 & 9/1   &  20 y 21  & El español de EEUU. Presentaciones.      \\
	\rowcolor[gray]{0.8}
			 & 14 & 10/1  &           & Examen final     \\
	\hline
\end{tabular*}
\end{center}
\end{document}